% Options for packages loaded elsewhere
% Options for packages loaded elsewhere
\PassOptionsToPackage{unicode}{hyperref}
\PassOptionsToPackage{hyphens}{url}
\PassOptionsToPackage{dvipsnames,svgnames,x11names}{xcolor}
%
\documentclass[
  letterpaper,
  DIV=11,
  numbers=noendperiod]{scrartcl}
\usepackage{xcolor}
\usepackage{amsmath,amssymb}
\setcounter{secnumdepth}{-\maxdimen} % remove section numbering
\usepackage{iftex}
\ifPDFTeX
  \usepackage[T1]{fontenc}
  \usepackage[utf8]{inputenc}
  \usepackage{textcomp} % provide euro and other symbols
\else % if luatex or xetex
  \usepackage{unicode-math} % this also loads fontspec
  \defaultfontfeatures{Scale=MatchLowercase}
  \defaultfontfeatures[\rmfamily]{Ligatures=TeX,Scale=1}
\fi
\usepackage{lmodern}
\ifPDFTeX\else
  % xetex/luatex font selection
\fi
% Use upquote if available, for straight quotes in verbatim environments
\IfFileExists{upquote.sty}{\usepackage{upquote}}{}
\IfFileExists{microtype.sty}{% use microtype if available
  \usepackage[]{microtype}
  \UseMicrotypeSet[protrusion]{basicmath} % disable protrusion for tt fonts
}{}
\makeatletter
\@ifundefined{KOMAClassName}{% if non-KOMA class
  \IfFileExists{parskip.sty}{%
    \usepackage{parskip}
  }{% else
    \setlength{\parindent}{0pt}
    \setlength{\parskip}{6pt plus 2pt minus 1pt}}
}{% if KOMA class
  \KOMAoptions{parskip=half}}
\makeatother
% Make \paragraph and \subparagraph free-standing
\makeatletter
\ifx\paragraph\undefined\else
  \let\oldparagraph\paragraph
  \renewcommand{\paragraph}{
    \@ifstar
      \xxxParagraphStar
      \xxxParagraphNoStar
  }
  \newcommand{\xxxParagraphStar}[1]{\oldparagraph*{#1}\mbox{}}
  \newcommand{\xxxParagraphNoStar}[1]{\oldparagraph{#1}\mbox{}}
\fi
\ifx\subparagraph\undefined\else
  \let\oldsubparagraph\subparagraph
  \renewcommand{\subparagraph}{
    \@ifstar
      \xxxSubParagraphStar
      \xxxSubParagraphNoStar
  }
  \newcommand{\xxxSubParagraphStar}[1]{\oldsubparagraph*{#1}\mbox{}}
  \newcommand{\xxxSubParagraphNoStar}[1]{\oldsubparagraph{#1}\mbox{}}
\fi
\makeatother

\usepackage{color}
\usepackage{fancyvrb}
\newcommand{\VerbBar}{|}
\newcommand{\VERB}{\Verb[commandchars=\\\{\}]}
\DefineVerbatimEnvironment{Highlighting}{Verbatim}{commandchars=\\\{\}}
% Add ',fontsize=\small' for more characters per line
\usepackage{framed}
\definecolor{shadecolor}{RGB}{241,243,245}
\newenvironment{Shaded}{\begin{snugshade}}{\end{snugshade}}
\newcommand{\AlertTok}[1]{\textcolor[rgb]{0.68,0.00,0.00}{#1}}
\newcommand{\AnnotationTok}[1]{\textcolor[rgb]{0.37,0.37,0.37}{#1}}
\newcommand{\AttributeTok}[1]{\textcolor[rgb]{0.40,0.45,0.13}{#1}}
\newcommand{\BaseNTok}[1]{\textcolor[rgb]{0.68,0.00,0.00}{#1}}
\newcommand{\BuiltInTok}[1]{\textcolor[rgb]{0.00,0.23,0.31}{#1}}
\newcommand{\CharTok}[1]{\textcolor[rgb]{0.13,0.47,0.30}{#1}}
\newcommand{\CommentTok}[1]{\textcolor[rgb]{0.37,0.37,0.37}{#1}}
\newcommand{\CommentVarTok}[1]{\textcolor[rgb]{0.37,0.37,0.37}{\textit{#1}}}
\newcommand{\ConstantTok}[1]{\textcolor[rgb]{0.56,0.35,0.01}{#1}}
\newcommand{\ControlFlowTok}[1]{\textcolor[rgb]{0.00,0.23,0.31}{\textbf{#1}}}
\newcommand{\DataTypeTok}[1]{\textcolor[rgb]{0.68,0.00,0.00}{#1}}
\newcommand{\DecValTok}[1]{\textcolor[rgb]{0.68,0.00,0.00}{#1}}
\newcommand{\DocumentationTok}[1]{\textcolor[rgb]{0.37,0.37,0.37}{\textit{#1}}}
\newcommand{\ErrorTok}[1]{\textcolor[rgb]{0.68,0.00,0.00}{#1}}
\newcommand{\ExtensionTok}[1]{\textcolor[rgb]{0.00,0.23,0.31}{#1}}
\newcommand{\FloatTok}[1]{\textcolor[rgb]{0.68,0.00,0.00}{#1}}
\newcommand{\FunctionTok}[1]{\textcolor[rgb]{0.28,0.35,0.67}{#1}}
\newcommand{\ImportTok}[1]{\textcolor[rgb]{0.00,0.46,0.62}{#1}}
\newcommand{\InformationTok}[1]{\textcolor[rgb]{0.37,0.37,0.37}{#1}}
\newcommand{\KeywordTok}[1]{\textcolor[rgb]{0.00,0.23,0.31}{\textbf{#1}}}
\newcommand{\NormalTok}[1]{\textcolor[rgb]{0.00,0.23,0.31}{#1}}
\newcommand{\OperatorTok}[1]{\textcolor[rgb]{0.37,0.37,0.37}{#1}}
\newcommand{\OtherTok}[1]{\textcolor[rgb]{0.00,0.23,0.31}{#1}}
\newcommand{\PreprocessorTok}[1]{\textcolor[rgb]{0.68,0.00,0.00}{#1}}
\newcommand{\RegionMarkerTok}[1]{\textcolor[rgb]{0.00,0.23,0.31}{#1}}
\newcommand{\SpecialCharTok}[1]{\textcolor[rgb]{0.37,0.37,0.37}{#1}}
\newcommand{\SpecialStringTok}[1]{\textcolor[rgb]{0.13,0.47,0.30}{#1}}
\newcommand{\StringTok}[1]{\textcolor[rgb]{0.13,0.47,0.30}{#1}}
\newcommand{\VariableTok}[1]{\textcolor[rgb]{0.07,0.07,0.07}{#1}}
\newcommand{\VerbatimStringTok}[1]{\textcolor[rgb]{0.13,0.47,0.30}{#1}}
\newcommand{\WarningTok}[1]{\textcolor[rgb]{0.37,0.37,0.37}{\textit{#1}}}

\usepackage{longtable,booktabs,array}
\usepackage{calc} % for calculating minipage widths
% Correct order of tables after \paragraph or \subparagraph
\usepackage{etoolbox}
\makeatletter
\patchcmd\longtable{\par}{\if@noskipsec\mbox{}\fi\par}{}{}
\makeatother
% Allow footnotes in longtable head/foot
\IfFileExists{footnotehyper.sty}{\usepackage{footnotehyper}}{\usepackage{footnote}}
\makesavenoteenv{longtable}
\usepackage{graphicx}
\makeatletter
\newsavebox\pandoc@box
\newcommand*\pandocbounded[1]{% scales image to fit in text height/width
  \sbox\pandoc@box{#1}%
  \Gscale@div\@tempa{\textheight}{\dimexpr\ht\pandoc@box+\dp\pandoc@box\relax}%
  \Gscale@div\@tempb{\linewidth}{\wd\pandoc@box}%
  \ifdim\@tempb\p@<\@tempa\p@\let\@tempa\@tempb\fi% select the smaller of both
  \ifdim\@tempa\p@<\p@\scalebox{\@tempa}{\usebox\pandoc@box}%
  \else\usebox{\pandoc@box}%
  \fi%
}
% Set default figure placement to htbp
\def\fps@figure{htbp}
\makeatother





\setlength{\emergencystretch}{3em} % prevent overfull lines

\providecommand{\tightlist}{%
  \setlength{\itemsep}{0pt}\setlength{\parskip}{0pt}}






\KOMAoption{captions}{tableheading}
\makeatletter
\@ifpackageloaded{caption}{}{\usepackage{caption}}
\AtBeginDocument{%
\ifdefined\contentsname
  \renewcommand*\contentsname{Table of contents}
\else
  \newcommand\contentsname{Table of contents}
\fi
\ifdefined\listfigurename
  \renewcommand*\listfigurename{List of Figures}
\else
  \newcommand\listfigurename{List of Figures}
\fi
\ifdefined\listtablename
  \renewcommand*\listtablename{List of Tables}
\else
  \newcommand\listtablename{List of Tables}
\fi
\ifdefined\figurename
  \renewcommand*\figurename{Figure}
\else
  \newcommand\figurename{Figure}
\fi
\ifdefined\tablename
  \renewcommand*\tablename{Table}
\else
  \newcommand\tablename{Table}
\fi
}
\@ifpackageloaded{float}{}{\usepackage{float}}
\floatstyle{ruled}
\@ifundefined{c@chapter}{\newfloat{codelisting}{h}{lop}}{\newfloat{codelisting}{h}{lop}[chapter]}
\floatname{codelisting}{Listing}
\newcommand*\listoflistings{\listof{codelisting}{List of Listings}}
\makeatother
\makeatletter
\makeatother
\makeatletter
\@ifpackageloaded{caption}{}{\usepackage{caption}}
\@ifpackageloaded{subcaption}{}{\usepackage{subcaption}}
\makeatother
\usepackage{bookmark}
\IfFileExists{xurl.sty}{\usepackage{xurl}}{} % add URL line breaks if available
\urlstyle{same}
\hypersetup{
  pdftitle={Trends in U.S. oyster landings},
  pdfauthor={Jorge Schmidt},
  colorlinks=true,
  linkcolor={blue},
  filecolor={Maroon},
  citecolor={Blue},
  urlcolor={Blue},
  pdfcreator={LaTeX via pandoc}}


\title{Trends in U.S. oyster landings}
\usepackage{etoolbox}
\makeatletter
\providecommand{\subtitle}[1]{% add subtitle to \maketitle
  \apptocmd{\@title}{\par {\large #1 \par}}{}{}
}
\makeatother
\subtitle{Volume and inflation-adjusted prices 1950 - 2024}
\author{Jorge Schmidt}
\date{2025-11-27}
\begin{document}
\maketitle


\subsection{Introduction}\label{introduction}

This repository contains data and code to build a dataset of landings of
fresh oysters from 1950 to 2024, inclusive, calculates weighed-average
inflation-adjusted prices, and breaks out data for the top five
producing states. It includes and aggregates all reported oyster species
in NOAA records.

\subsection{Objectives}\label{objectives}

It is intended to support decision-making in the oyster production and
distribution sectors by displaying trends in volumes and prices.

\subsection{Data Sources}\label{data-sources}

The data for oyster landings volumes and historical prices comes from
NOAA. The data to calculate the adjustments for inflation comes from the
St.~Louis Fed (FRED).

\subsection{Data Processing}\label{data-processing}

\begin{enumerate}
\def\labelenumi{\arabic{enumi}.}
\item
  The file from NOAA contains 1,767 yearly observations (rows) and 11
  columns (year + 10 variables). Some of the data is redundant (common
  and scientific names, and tsn), adds imprecision (metric tons and
  lbs), or is irrelevant for the purposes of this analysis (collection,
  confidentiality, source).
\item
  The file from FRED contains 908 monthly observations (rows) and two
  columns (date and CPI value).
\item
  Initial processing included cleaning up file names, filtering,
  grouping and/or summarizing, and exporting the data.
\item
  Further processing included plotting data and a spatial visualization.
\end{enumerate}

\subsection{Main Findings}\label{main-findings}

\section{Volumes}\label{volumes}

Oyster landings in the U.S. have declined from a maximum of 82.2 million
pounds in 1952 to 19.4 million pounds in 2024.

Figure~\ref{fig-landingsvol} shows the volume trend.

\begin{Shaded}
\begin{Highlighting}[]
\InformationTok{\textasciigrave{}\textasciigrave{}\textasciigrave{}\{r\}}
\CommentTok{\#| label: fig{-}landingsvol}
\CommentTok{\#| fig{-}cap: "Trend in U.S. oyster landings"}
\CommentTok{\#| warning: false}
\CommentTok{\#| code{-}fold: true}
\CommentTok{\#| code{-}summary: "Show code"}

\FunctionTok{library}\NormalTok{(tidyverse)}
\FunctionTok{library}\NormalTok{(ggridges)}
\FunctionTok{library}\NormalTok{(readr)}
\FunctionTok{library}\NormalTok{(ggplot2)}
\FunctionTok{library}\NormalTok{(dplyr)}
\FunctionTok{library}\NormalTok{(ggimage)}
\FunctionTok{library}\NormalTok{(magick)}
\FunctionTok{library}\NormalTok{(here)}

\NormalTok{landings }\OtherTok{\textless{}{-}} \FunctionTok{read\_rds}\NormalTok{(}\StringTok{"data/output/landings\_inflation\_adjusted.rds"}\NormalTok{)}

\CommentTok{\# create data frame combining landings and oyster imagea}
\NormalTok{landings}\SpecialCharTok{$}\NormalTok{img }\OtherTok{\textless{}{-}} \FunctionTok{rep}\NormalTok{(}\StringTok{"data/processed/oyster\_resized.png"}\NormalTok{, }\FunctionTok{nrow}\NormalTok{(landings))}

\CommentTok{\# plot}
\FunctionTok{ggplot}\NormalTok{(}
  \AttributeTok{data =}\NormalTok{ landings,}
  \AttributeTok{mapping =} \FunctionTok{aes}\NormalTok{(}\AttributeTok{x =}\NormalTok{ year,}
                \AttributeTok{y =}\NormalTok{ total\_pounds}\SpecialCharTok{/}\DecValTok{2204}\NormalTok{)) }\SpecialCharTok{+}
  \FunctionTok{geom\_image}\NormalTok{(}\FunctionTok{aes}\NormalTok{(}\AttributeTok{image =}\NormalTok{ img), }\AttributeTok{size =}\NormalTok{ landings}\SpecialCharTok{$}\NormalTok{total\_pounds}\SpecialCharTok{/}\DecValTok{1500000000}\NormalTok{) }\SpecialCharTok{+}
  \FunctionTok{labs}\NormalTok{(}
  \AttributeTok{title =} \StringTok{"Landings of fresh oysters in the U.S. (1950 {-} 2024)"}\NormalTok{,}
  \AttributeTok{subtitle =} \StringTok{"Expressed in weight of oyster meat"}\NormalTok{,}
  \AttributeTok{x =} \StringTok{"Year"}\NormalTok{,}
  \AttributeTok{y =} \StringTok{"Metric tons"}\NormalTok{,}
  \AttributeTok{caption =} \StringTok{"Data from NOAA"}\NormalTok{) }\SpecialCharTok{+}
  \FunctionTok{scale\_y\_continuous}\NormalTok{(}\AttributeTok{limits =} \FunctionTok{c}\NormalTok{(}\DecValTok{0}\NormalTok{, }\FunctionTok{max}\NormalTok{(landings}\SpecialCharTok{$}\NormalTok{total\_pounds}\SpecialCharTok{/}\DecValTok{2204}\NormalTok{))) }\SpecialCharTok{+}
  \FunctionTok{theme}\NormalTok{(}
  \AttributeTok{caption.justification =} \StringTok{"right"}\NormalTok{,}
  \AttributeTok{legend.position =} \StringTok{"bottom"}\NormalTok{,}
  \AttributeTok{legend.justification =} \StringTok{"center"}\NormalTok{) }\SpecialCharTok{+}
  \FunctionTok{scale\_color\_hue}\NormalTok{(}\AttributeTok{labels =} \FunctionTok{c}\NormalTok{(}\StringTok{"Landings"}\NormalTok{, }\StringTok{"Imports"}\NormalTok{)) }\SpecialCharTok{+}
  \FunctionTok{theme\_minimal}\NormalTok{()}
\InformationTok{\textasciigrave{}\textasciigrave{}\textasciigrave{}}
\end{Highlighting}
\end{Shaded}

\begin{figure}[H]

\centering{

\pandocbounded{\includegraphics[keepaspectratio]{final_report_files/figure-pdf/fig-landingsvol-1.pdf}}

}

\caption{\label{fig-landingsvol}Trend in U.S. oyster landings}

\end{figure}%

\section{Inflation-adjusted price of oyster
meat}\label{inflation-adjusted-price-of-oyster-meat}

For many decades and most of the time period studied, the
inflation-adjusted price of oyster meat fluctuated between \$5.00 and
\$7.50. In 2013, the price began to rise significantly from this
previous range and reached a high of \$12.10 in 2023.

Figure~\ref{fig-landingsprice} shows the price trend.

\begin{Shaded}
\begin{Highlighting}[]
\InformationTok{\textasciigrave{}\textasciigrave{}\textasciigrave{}\{r\}}
\CommentTok{\#| label: fig{-}landingsprice}
\CommentTok{\#| fig{-}cap: "Trend in U.S. oyster meat price"}
\CommentTok{\#| warning: false}
\CommentTok{\#| code{-}fold: true}
\CommentTok{\#| code{-}summary: "Show code"}

\CommentTok{\# plot}
\FunctionTok{ggplot}\NormalTok{(}
  \AttributeTok{data =}\NormalTok{ landings,}
  \AttributeTok{mapping =} \FunctionTok{aes}\NormalTok{(}\AttributeTok{x =}\NormalTok{ year,}
                \AttributeTok{y =}\NormalTok{ adj\_dollars)) }\SpecialCharTok{+}
  \FunctionTok{geom\_image}\NormalTok{(}\FunctionTok{aes}\NormalTok{(}\AttributeTok{image =}\NormalTok{ img), }\AttributeTok{size =}\NormalTok{ landings}\SpecialCharTok{$}\NormalTok{adj\_dollars}\SpecialCharTok{/}\DecValTok{200}\NormalTok{) }\SpecialCharTok{+}
  \FunctionTok{labs}\NormalTok{(}
  \AttributeTok{title =} \StringTok{"Landings of fresh oysters in the U.S. (1950 {-} 2024)"}\NormalTok{,}
  \AttributeTok{subtitle =} \StringTok{"In 2025 inflation{-}adjusted dollars per pound of oyster meat"}\NormalTok{,}
  \AttributeTok{x =} \StringTok{"Year"}\NormalTok{,}
  \AttributeTok{y =} \StringTok{"Dollars"}\NormalTok{,}
  \AttributeTok{caption =} \StringTok{"Data from NOAA and FRED"}\NormalTok{) }\SpecialCharTok{+}
  \FunctionTok{scale\_y\_continuous}\NormalTok{(}\AttributeTok{limits =} \FunctionTok{c}\NormalTok{(}\DecValTok{0}\NormalTok{, }\DecValTok{13}\NormalTok{)) }\SpecialCharTok{+}
  \FunctionTok{theme}\NormalTok{(}
  \AttributeTok{caption.justification =} \StringTok{"right"}\NormalTok{,}
  \AttributeTok{legend.position =} \StringTok{"bottom"}\NormalTok{,}
  \AttributeTok{legend.justification =} \StringTok{"center"}\NormalTok{) }\SpecialCharTok{+}
  \FunctionTok{theme\_minimal}\NormalTok{()}
\InformationTok{\textasciigrave{}\textasciigrave{}\textasciigrave{}}
\end{Highlighting}
\end{Shaded}

\begin{figure}[H]

\centering{

\pandocbounded{\includegraphics[keepaspectratio]{final_report_files/figure-pdf/fig-landingsprice-1.pdf}}

}

\caption{\label{fig-landingsprice}Trend in U.S. oyster meat price}

\end{figure}%

\section{Top five producing U.S.
states}\label{top-five-producing-u.s.-states}

From 1950 through 2024, the top five oyster producing states, as
measured by inflation-adjusted value of the landings, were (in
descending order): Louisiana, Maryland, Virginia, Washington, and Texas.
The most consistent producers over this time period were Washington and
Louisiana. Texas, Virginia, and Maryland all had gaps with no reported
production.

Figure~\ref{fig-topfive} shows the comparison.

\begin{Shaded}
\begin{Highlighting}[]
\InformationTok{\textasciigrave{}\textasciigrave{}\textasciigrave{}\{r\}}
\CommentTok{\#| label: fig{-}topfive}
\CommentTok{\#| fig{-}cap: "Top five U.S. states"}
\CommentTok{\#| warning: false}
\CommentTok{\#| code{-}fold: true}
\CommentTok{\#| code{-}summary: "Show code"}

\CommentTok{\# load data}
\NormalTok{top\_states }\OtherTok{\textless{}{-}} \FunctionTok{read\_rds}\NormalTok{(}\StringTok{"data/output/landings\_by\_top\_state\_inflation\_adjusted "}\NormalTok{)}

\DocumentationTok{\#\# define states as factors}
\NormalTok{top\_states }\OtherTok{\textless{}{-}}\NormalTok{ top\_states }\SpecialCharTok{|\textgreater{}}
  \FunctionTok{mutate}\NormalTok{(}\AttributeTok{state =} \FunctionTok{factor}\NormalTok{(state,}
                        \AttributeTok{levels =} \FunctionTok{c}\NormalTok{(}\StringTok{"Other"}\NormalTok{,        }\CommentTok{\# bottom}
                                   \StringTok{"Louisiana"}\NormalTok{,}
                                   \StringTok{"Maryland"}\NormalTok{,}
                                   \StringTok{"Virginia"}\NormalTok{,}
                                   \StringTok{"Washington"}\NormalTok{,}
                                   \StringTok{"Texas"}\NormalTok{)))   }\CommentTok{\# top}

\CommentTok{\# plot}
\FunctionTok{ggplot}\NormalTok{(}
  \AttributeTok{data =}\NormalTok{ top\_states,}
  \AttributeTok{mapping =} \FunctionTok{aes}\NormalTok{(}\AttributeTok{x =}\NormalTok{ year,}
             \AttributeTok{y =}\NormalTok{ state,}
             \AttributeTok{height =}\NormalTok{ adj\_dollars }\SpecialCharTok{/} \FloatTok{1e6}\NormalTok{,}
             \AttributeTok{fill =}\NormalTok{ state)) }\SpecialCharTok{+}
  \FunctionTok{geom\_density\_ridges}\NormalTok{(}\AttributeTok{stat =} \StringTok{"identity"}\NormalTok{,   }
                      \AttributeTok{scale =} \FloatTok{0.95}\NormalTok{,}
                      \AttributeTok{alpha =} \FloatTok{0.99}\NormalTok{,}
                      \AttributeTok{color =} \StringTok{"white"}\NormalTok{) }\SpecialCharTok{+}
  \FunctionTok{scale\_fill\_manual}\NormalTok{(}\AttributeTok{values =} \FunctionTok{c}\NormalTok{(}\StringTok{"gray25"}\NormalTok{, }\StringTok{"gray25"}\NormalTok{, }\StringTok{"gray25"}\NormalTok{,}
                               \StringTok{"gray25"}\NormalTok{, }\StringTok{"gray25"}\NormalTok{, }\StringTok{"gray25"}\NormalTok{)) }\SpecialCharTok{+}
  \FunctionTok{labs}\NormalTok{(}\AttributeTok{title =} \StringTok{"Value of oyster production top five U.S. states (1950 {-} 2024)"}\NormalTok{,}
       \AttributeTok{subtitle =} \StringTok{"In inflation{-}adjusted 2025 dollars"}\NormalTok{,}
       \AttributeTok{x =} \StringTok{"Year"}\NormalTok{,}
       \AttributeTok{y =} \ConstantTok{NULL}\NormalTok{,}
       \AttributeTok{caption =} \StringTok{"Data from NOAA and FRED"}\NormalTok{) }\SpecialCharTok{+}
  \FunctionTok{theme\_ridges}\NormalTok{(}\AttributeTok{grid =} \ConstantTok{FALSE}\NormalTok{,}
               \AttributeTok{font\_size =} \DecValTok{13}\NormalTok{) }\SpecialCharTok{+}
  \FunctionTok{theme\_minimal}\NormalTok{() }\SpecialCharTok{+}
  \FunctionTok{theme}\NormalTok{(}\AttributeTok{legend.position =} \StringTok{"none"}\NormalTok{,}
        \AttributeTok{plot.title =} \FunctionTok{element\_text}\NormalTok{(}\AttributeTok{face =} \StringTok{"bold"}\NormalTok{))}
\InformationTok{\textasciigrave{}\textasciigrave{}\textasciigrave{}}
\end{Highlighting}
\end{Shaded}

\begin{figure}[H]

\centering{

\pandocbounded{\includegraphics[keepaspectratio]{final_report_files/figure-pdf/fig-topfive-1.pdf}}

}

\caption{\label{fig-topfive}Top five U.S. states}

\end{figure}%

\section{A nice map}\label{a-nice-map}

This map of the 48 contiguous united states highlights the importance of
Louisiana, Maryland, Virginia, Washington, and Texas to U.S. oyster
production. \begin{center}
\includegraphics[width=1\linewidth,height=\textheight,keepaspectratio]{results/img/oyster_landing_by_state.png}
\end{center}

\subsection{References}\label{references}

\begin{itemize}
\item
  The landings data {[}FOSS\_landings.xlsx{]} was obtained from
  https://www.fisheries.noaa.gov/foss/f?p=215:200:7482903932446
\item
  The inflation data {[}CPIAUCSL.csv{]} was obtained from
  https://fred.stlouisfed.org/series/CPIAUCSL
\end{itemize}




\end{document}
